%----------------------------------------------------------------------------------------
%	Metropolia Thesis LaTeX Template
%----------------------------------------------------------------------------------------
% License:
% This work is licensed under the Creative Commons Attribution 4.0 International License.
% To view a copy of this license, visit http://creativecommons.org/licenses/by/4.0/.
%
% However, this license apply to this template. As a template, it is supposed to be
% modified for your own needs (with your thesis content). For this reason, if you use
% this project as a template and not specifically distribute it as part of a another
% package/program, we grant the extra permission to freely copy and modify these files as
% you see fit and even to delete this copyright notice.
% In short, you are free to publish your thesis under whatever license you wish, even
% keep the all rights reserved to you.
%
% Authors:
% Panu Leppäniemi, Patrik Luoto, Mikaa Oni and Patrick Ausderau
%
% Credits:
% Panu Leppäniemi: abstract, def, cleaning,...
% Patrik Luoto: title page, abstract in Finnish, abbreviation, math,...
% Mikaa Oni: switch to biber biblatex
% Patrick Ausderau: initial version, style, table of content, bibliography, figure,
%                   appendix, table, source code listing,...
%
% Please:
% If you find mistakes, improve this template and alike, please contribute by sharing
% your improvements and/or send us your feedback there:
% https://github.com/panunu/metropolia-thesis-latex
% And of course, if you improve it, add yourself as an author.
%
% Compiler:
% Use XeLaTeX as a compiler. LuaLaTeX works too.
% Typical compilation:
% # minted require -shell-escape to run  external script.
% # -8bit avoid ^^I for tabs in minted.
% $ xelatex -shell-escape -8bit main
% # If any change in the bibliography
% $ biber main
% # If any change with the abbreviation or acronym
% $ makeglossaries main
% #Then compile again
% $ xelatex -shell-escape -8bit main
% #And if still some citation or label warnings, compile once more
% $ xelatex -shell-escape -8bit main

%----------------------------------------------------------------------------------------
%	THESIS INFO
%----------------------------------------------------------------------------------------

% All general information (main language, title, author (you), degree programme, major
% option, etc.)
% Edit the file chapters/0info.tex to change these information
\documentclass[12pt,a4paper,oneside,article]{memoir}%Do not touch this first line ;)

% Global information (title of your thesis, your name, degree programme, major, etc.)

\def\bilingual{no}%For international student writing in English, only one language and one abstract.
%\def\bilingual{yes}%For Finnish students, you must have 2 abstracts, one in English and one in your native language (Finnish or Swedish), so "yes", your thesis is bilingual.

\def\thesislang{english} % "english" is the only other supported language currently. If someone has the swedish, please contribute!
%\def\thesislang{finnish} %change this depending on the main language of the thesis.

\def\secondlang{english} %if the main language is Finnish (or Swedish), you must have 2 abstracts (one in Finnish (or Swedish) and one in English)
%\def\secondlang{finnish}
%If the main language is English and that you are native Finnish (or Swedish) speaker, you must have also abstract in your native language on top of the English one.

\author{Hai To} %your first name and last name

%\def\alaotsikko{Alaotsikko/Subtitle} %DISABLED, seems not to be an option with the 2018 template. If you really need it, uncomment and modify style/title.tex accordingly.

%License
%When publishing your thesis to theseus.fi, you can keep all rights reserved to you,
%or use one of the Creative Commons https://creativecommons.org/licenses/?lang=en
%This attribute will set the license in the metadata of the generated pdf.
%possible options (case sensitive):
%all (keep all rights reserved to yourself)
%by (Attribution)
%by-sa (Attribution-ShareAlike)
%by-nd (Attribution-NoDerivs)
%by-nc (Attribution-NonCommercial)
%by-nc-sa (Attribution-NonCommercial-ShareAlike)
%by-nc-nd (Attribution-NonCommercial-NoDerivs)
%Note that theseus do not accept dedication to public domain CC0
\def\thesiscopy{all}

%Finnish section, for title/abstract
\def\otsikko{Sensorien ja simulaatioympäristön integrointi liikkuvalle robottialustalle}
\def\tutkinto{Insinööri (YAMK)} % Master of Engineering
\def\kohjelma{Tieto- ja viestintätekniikka}
\def\suuntautumis{Verkot ja palvelut}
\def\thesisfi{Opinnäytetyö}
\def\ohjaajat{
Markku Niiranen, Yliopettaja\newline
}
\def\tiivistelma{
Opinnäytetyö keskittyy kattavan digitaalisen kaksosen kehittämiseen liikkuvalle robottialustalle sensorien integroinnin ja simulaatioympäristön kautta. Työn pääpaino on luoda tarkka digitaalinen malli, joka heijastaa fyysisen robotin ominaisuuksia ja käyttäytymistä. Työ jakautuu kolmeen vaiheeseen: sensorien integrointi, digitaalisen kaksosen kehittäminen ja validointi. \newline

Tutkimus käyttää ROS2:ta reaaliaikaisen tiedonsiirron hallintaan fyysisen ja digitaalisen järjestelmän välillä. Digitaalinen kaksonen sisältää LiDAR-, kamera- ja IMU-sensorit sekä fysiikkapohjaiset simulaatiot. Odotetut tulokset sisältävät validoidun metodologian digitaalisten kaksosten luomiseen, toimivan cyber-fyysisen järjestelmän ja virtuaalisen testausympäristön robotiikkatutkimukselle.
}
\def\avainsanat{Digitaalinen kaksonen, Sensorien integrointi, Simulaatio, ROS2, Cyber-fyysinen järjestelmä}
\def\aihe{Digitaalisen kaksosen kehittäminen liikkuvalle robottialustalle sensorien integroinnin ja simulaatioympäristön kautta.}%for the pdf metadata/properties. If not used, empty it and also the \def\subject.

%English section, for title/abstract
\title{Integration of Sensors and Simulation Environment for a Mobile Robot Platform}
\def\metropoliadegree{Master of Engineering} % change to your needs, e.g. "master", etc.
\def\metropoliadegreeprogramme{Information Technology}
\def\metropoliaspecialisation{Networking and Services}
\def\thesisen{Master's Thesis} % change to your need, e.g. master's
\def\metropoliainstructors{
Markku Niiranen, Principal Lecturer\newline
}
\def\abstract{
The thesis focuses on developing a comprehensive digital twin for a mobile robot platform through sensor integration and simulation environment development. The primary objective is creating an accurate digital representation that mirrors the physical robot's capabilities and behaviors. The work is divided into three phases: sensor integration, digital twin development, and validation. \newline

The study utilizes ROS2 for real-time communication between physical and digital systems. The digital twin incorporates LiDAR, camera, and IMU sensors with physics-based simulations to ensure high fidelity. Expected outcomes include a validated methodology for digital twin creation, a functioning cyber-physical system, and a virtual testing environment that advances mobile robotics research and development.
}
\def\metropoliakeywords{Digital Twin, Sensor Integration, Simulation, ROS2, Cyber-Physical Systems}
\def\subject{Digital twin development for mobile robot platform through sensor integration and simulation environment.}%for the pdf metadata/properties. If not used, empty it and also the \def\aihe.his first line ;)


%----------------------------------------------------------------------------------------
%	GLOBAL STYLES
%----------------------------------------------------------------------------------------

% If you need extra package, etc. modify the style/style.tex file.
% If you are using Windows OS, you will need to change default font to Arial in that
% style/style.tex file (or install Liberation Sans font to your system).
% If you are using MacOS or linux, make sure you have Liberation Sans font installed.
\input{style/style.tex}
% Normally, you do not need to modify the title style. It's content comes from the
% chapters/0info.tex file.
\input{style/title.tex}

%----------------------------------------------------------------------------------------
%	ABBREVIATION AND GLOSSARY
%----------------------------------------------------------------------------------------

% Add/edit all your acronyms, abbreviations, glossary entries, etc. definitions in
% chapters/0abbr.tex file.
% You can have as many as you wish. Only the ones you use in your text (inserted with
% \gls{} command) will print in the Glossary/Lyhenteet.
% Generate the glossary
\makeglossaries

% Acronyms, abbreviations, etc.

\newacronym{amr}{AMR}{Autonomous Mobile Robot}
\newacronym{lidar}{LiDAR}{Light Detection and Ranging}
\newacronym{imu}{IMU}{Inertial Measurement Unit}
\newacronym{ros2}{ROS2}{Robot Operating System 2}
\newacronym{ros}{ROS}{Robot Operating System}
\newacronym{slam}{SLAM}{Simultaneous Localization and Mapping}
\newacronym{urdf}{URDF}{Unified Robot Description Format}
\newacronym{nav2}{NAV2}{ROS2 Navigation Stack}
\newacronym{sme}{SME}{Small and Medium-sized Enterprise}
\newacronym{ai}{AI}{Artificial Intelligence}
\newacronym{modbus}{MODBUS}{Modular Bus Protocol}
\newacronym{gazebo}{Gazebo}{Robot Simulation Platform}
\newacronym{isaac}{Isaac Sim}{NVIDIA Isaac Simulation Platform}

% Glossary entries

\newglossaryentry{techboost}{
	name={TECHBOOST project},
	description={Research project on mobile robots that helps small and medium-sized enterprises (SMEs) with significant growth potential to adopt, develop and systematically apply new technologies, especially in robotics and artificial intelligence (AI)}
}

\newglossaryentry{jaska}{
	name={Jaska},
	description={Metropolia's mobile robot platform used for research and development in autonomous navigation and sensor integration}
}

\newglossaryentry{pointlio}{
	name={POINT-LIO},
	description={A lightweight LiDAR-Inertial Odometry algorithm designed for efficient real-time SLAM on resource-constrained robotic platforms}
}

\newglossaryentry{sensorfusion}{
	name={sensor fusion},
	description={The process of combining data from multiple sensors to produce more accurate, reliable, and useful information than could be obtained from any individual sensor alone}
}




%----------------------------------------------------------------------------------------
%	DOCUMENT STARTS HERE...
%----------------------------------------------------------------------------------------

\begin{document}
\IfLanguageName{finnish}{
}{
  \raggedright%2021 template, align left, no hyphennization for English version
}
\counterwithout{listing}{chapter}

%----------------------------------------------------------------------------------------
%	TITLE PAGE
%----------------------------------------------------------------------------------------

\input{style/title_headers.tex}
\maketitle
\newpage

%----------------------------------------------------------------------------------------
%	ABSTRACT / Tiivistelmä
%----------------------------------------------------------------------------------------

% If you are international student writing in English, ignore the Finnish abstract.
% If you are Finnish citizen, you must have 2 abstracts, one in Finnish (or Swedish
% depending on your mother tongue) and one in English regardless of the main language of
% your thesis. Normally, you do not need to modify the abstract style. It's content comes
% from the chapters/0info.tex file.
\ifdefstring{\bilingual}{no}{%
    \input{style/abstract_en.tex}
    }{%
    \IfLanguageName{finnish}{%order of abstracts based on main language and spacing hell
        \input{style/abstract_fi_fi.tex}
        \input{style/abstract_fi_en.tex}
        }{
        \input{style/abstract_en.tex}
        \input{style/abstract_fi.tex}
    }
}
%----------------------------------------------------------------------------------------
%	License? Acknowledgement?
%----------------------------------------------------------------------------------------

% Uncomment next line and edit chapters/0license.tex if you want license in your thesis.
%\input{chapters/0license.tex}

% Uncomment next line and edit chapters/0acknowledgement.tex if you want acknowledgements.
%% Acknowledgement
% If relevant give a special thanks to the people who supported you during your thesis
% writing.

\pagestyle{empty}
\chapter*{Acknowledgement}

Thanks to Ngoc Nguyen for proofreading this thesis.

Thanks to Panu Leppäniemifor the \LaTeX{} ~thesis template.

\clearpage



%----------------------------------------------------------------------------------------
%	TABLE OF CONTENTS
%----------------------------------------------------------------------------------------

\input{style/toc.tex}

%list of figure, tables would come here if relevant?

%----------------------------------------------------------------------------------------
%	Lyhenteet / Abbreviation
%----------------------------------------------------------------------------------------

% If you don't use abbreviations/glossary, remove the following line.
\input{style/abbr.tex}

%----------------------------------------------------------------------------------------
%	CONTENT
%----------------------------------------------------------------------------------------

\input{style/content.tex}%reset page number to 1, etc.

% Thesis content if you strictly follow the "Final Year Project guide". Of course, you
% can adapt to your specific needs (add more chapter, rename them, etc.).
% Introduction

\chapter{Introduction}

\section{Background and Context}

The need for advanced autonomous systems is growing quickly, making mobile robotics
an important area of study, especially for tough industrial settings like factories
or mines. This Master's Thesis focuses on two key needs: combining sensors effectively
and creating reliable simulations. These are essential for building and testing
advanced robotic systems that can work on their own. Challenges include dealing with
sensor errors, changing environments, and the need for efficient testing methods.

The main focus is a mobile robot used as a prototype to test autonomous navigation,
sensor integration, and new technologies like custom wheels and improved LiDAR mapping.
A big first step is getting this robot working again so it can be used for experiments.
This involves connecting different sensors—such as LiDAR, cameras, and Inertial
Measurement Units (IMUs)—into one system. This can be hard due to issues like
mismatched data and environmental noise. Simulations are also crucial to test the
robot virtually before real-world use, saving time and money.

This study builds on existing research. Quigley et al. (2009) developed the Robot
Operating System (ROS), a widely used tool that helps combine sensors for autonomous
robots. Thrun et al. (2005) explain how probability can improve location and mapping,
showing the importance of good sensor fusion. Siciliano et al. (2009) stress the
value of simulations to connect theory with practice. These studies highlight gaps
in real-time sensor use and simulation accuracy, which this thesis aims to address.

To do this work, the researcher needs specific skills. A strong background in
mechatronics, automation, robotics, or electrical engineering is necessary. Good
knowledge of programming languages like Python and C++ is required for writing
control codes and sensor programs. Familiarity with robotics tools like ROS/ROS2
is also helpful, as it makes the work easier and fits industry standards.

\section{Thesis Objectives and Scope}

The main goal of this thesis is to integrate sensors and a simulation environment
for a mobile robot platform, creating a system that can operate autonomously.
This means carefully adding and testing important parts like motors, cameras,
LiDAR, and IMUs to ensure they work well together. The work is divided into
three clear phases, moving from hardware setup to digital testing and real-world checks.

\textbf{Phase 1: Hardware Foundation and Initial Perception}
This phase sets up the robot's physical parts and collects basic environment data.
Tasks include adding motor controllers, a camera, LiDAR, and IMU to the robot.
This involves solving technical problems, like connecting the Unitree L2 LiDAR driver
and setting up a joystick with a MODBUS motor driver for manual control. For perception,
the phase includes making a map using POINT-LIO SLAM, a simple mapping tool for robots
with limited power. It also involves reviewing simulation tools—Gazebo, Webots,
CoppeliaSim, and Isaac Sim—to pick the best one for long-term use, based on ROS2
compatibility as noted by Quigley et al. (2009).

\textbf{Phase 2: Modeling and Simulation Implementation}
Phase 2 focuses on building a digital twin of the robot and testing the chosen
simulation tool. A key task is creating a detailed Jaska model that matches the
real robot's parts (base, LiDAR, camera, IMU). This model is used to test mapping
and location in Gazebo and Isaac Sim, using probability methods from Thrun et al. (2005)
to mimic real conditions. The results help choose the best simulation tool, which
is then set up with the robot's hardware to improve development.

\textbf{Phase 3: Autonomous Navigation and Validation}
The final phase builds and tests a complete software system for autonomy in the
real world. This requires studying ROS2, an updated version of ROS designed for
distributed systems, as explained by Quigley et al. (2009). The focus is on using
ROS2 tools like NAV2 for navigation and avoiding obstacles. The main task is creating
a ROS2 software system that handles sensor data, accurate location, and obstacle
avoidance. Testing includes checking sensor fusion for navigation, leading to
real-world demonstrations of autonomous movement, supported by control ideas from
Siciliano et al. (2009).

\section{Expected Outcomes and Contributions}

Completing this thesis will provide important results for mobile robotics. First,
it will deliver a fully working robot with integrated sensors and motor controls,
ready for further tests. Second, it will set a standard simulation tool for future
research, addressing the need for reliable virtual testing as suggested by
Siciliano et al. (2009). Third, a basic ROS2 system will enable the robot to navigate
on its own, advancing sensor-based autonomy as supported by Thrun et al. (2005).
These achievements will improve the robot and add valuable knowledge to the field
of mobile robotics.

% Theoretical Framework and Technology Review
%\clearpage%if the chapter heading starts close to bottom of the page, force a line break and remove the vertical vspace
\vspace{21.5pt}
\chapter{Theoretical Framework and Technology Review}

\section{Digital Twin Technology Fundamentals}

[Content to be added - digital twin concepts, cyber-physical systems, real-time synchronization principles, digital twin architectures]

\section{Sensor Integration for Digital Twins}

[Content to be added - multi-sensor data fusion, sensor calibration, synchronization techniques for digital twin applications]

\subsection{LiDAR Technology and 3D Environment Mapping}

[Content to be added - LiDAR principles for digital twin creation, point cloud processing, 3D reconstruction]

\subsection{Computer Vision and Camera Integration}

[Content to be added - camera calibration, visual perception for digital twins, image processing techniques]

\subsection{Inertial Measurement Units (IMU) and Motion Tracking}

[Content to be added - IMU principles, motion tracking, pose estimation for digital twin synchronization]

\section{The Robot Operating System 2 (ROS2) for Cyber-Physical Systems}

[Content to be added - ROS2 architecture for digital twins, real-time communication, distributed systems for cyber-physical applications]

\section{Simulation Environments and Physics Engines}

[Content to be added - comparison of simulation platforms (Gazebo, Isaac Sim, Webots), physics engines, rendering for digital twins]

\subsection{Inertial Measurement Units (IMU) and Sensor Fusion}

[Content to be added - IMU principles, sensor fusion techniques, Kalman filtering]

\section{Review of Robotic Simulation Environments}

[Content to be added - overview of simulation in robotics, importance, challenges]

\subsection{Gazebo}

[Content to be added - Gazebo capabilities, integration with ROS]

\subsection{Isaac Sim}

[Content to be added - NVIDIA Isaac Sim features, advantages]

\subsection{Other Relevant Simulators (Webots, CoppeliaSim)}

[Content to be added - brief overview of alternative simulation platforms]

% System Design and Hardware Integration
%\clearpage%if the chapter heading starts close to bottom of the page, force a line break and remove the vertical vspace
\vspace{21.5pt}
\chapter{System Design and Hardware Integration}

\section{The ``Jaska'' Mobile Robot Platform Architecture}

[Content to be added - description of the Jaska robot platform, specifications, capabilities]

\section{Motor Control System and Driver Integration (MODBUS)}

[Content to be added - motor control systems, MODBUS protocol implementation, joystick integration]

\section{Sensor Suite Integration}

[Content to be added - overview of sensor integration approach, challenges, solutions]

\subsection{Unitree L2 LiDAR Integration}

[Content to be added - specific details about Unitree L2 LiDAR integration, driver development]

\subsection{Stereo Camera and IMU Setup}

[Content to be added - camera and IMU integration, calibration procedures]

\section{Jaska Platform URDF Model Creation}

[Content to be added - URDF model development, joint definitions, sensor placement]

% Simulation Environment Implementation and ROS2 Software Stack
%\clearpage%if the chapter heading starts close to bottom of the page, force a line break and remove the vertical vspace
\vspace{21.5pt}
\chapter{Simulation Environment Implementation}

\section{Evaluation and Selection of a Simulation Environment}

[Content to be added - comparison criteria, evaluation results, selection rationale]

\section{Configuration of the Chosen Simulator (e.g., Isaac Sim)}

[Content to be added - detailed setup and configuration of the selected simulator]

\section{Mapping and Localization Validation in Simulation}

[Content to be added - simulation-based validation of SLAM algorithms]

\vspace{21.5pt}
\chapter{ROS2 Software Stack Implementation}

\section{Development of ROS2 Sensor and Actuator Drivers}

[Content to be added - ROS2 driver development, node architecture]

\section{Implementation of Localization and Mapping}

[Content to be added - SLAM implementation in ROS2, sensor fusion]

\section{Configuration of the ROS2 Navigation Stack (NAV2)}

[Content to be added - NAV2 setup, configuration, parameter tuning]

\section{Obstacle Avoidance and Path Planning}

[Content to be added - path planning algorithms, obstacle detection and avoidance]

% Experimental Results and Evaluation
%\clearpage%if the chapter heading starts close to bottom of the page, force a line break and remove the vertical vspace
\vspace{21.5pt}
\chapter{Experimental Results and Evaluation}

\section{Sensor Validation and System Integration Tests}

[Content to be added - test results for sensor integration, system performance metrics]

\section{Performance of Mapping and Localization in the Real World}

[Content to be added - real-world SLAM performance, accuracy measurements, comparison with simulation]

\section{Evaluation of Autonomous Navigation Proof-of-Concept}

[Content to be added - autonomous navigation test results, path planning accuracy, obstacle avoidance performance]

% Discussion and Future Work
%\clearpage%if the chapter heading starts close to bottom of the page, force a line break and remove the vertical vspace
\vspace{21.5pt}
\chapter{Discussion and Future Work}

\section{Analysis of Achieved Outcomes vs. Objectives}

[Content to be added - comparison of achieved results with initial objectives, success metrics]

\section{Challenges Encountered and Lessons Learned}

[Content to be added - technical challenges, solutions implemented, lessons for future projects]

\section{Recommendations for Future Development}

[Content to be added - suggestions for improvement, next steps, potential research directions]

% Conclusion
%\clearpage%if the chapter heading starts close to bottom of the page, force a line break and remove the vertical vspace
\vspace{21.5pt}
\chapter{Conclusion}

This thesis successfully developed a comprehensive digital twin system for a mobile robot platform, achieving the primary objective of creating an accurate digital representation through systematic sensor integration and simulation environment implementation. The research demonstrated that high-fidelity digital twins can be effectively created for mobile robotic systems, providing valuable tools for virtual testing, development, and validation.

The three-phase approach proved effective: sensor integration established the foundation for data acquisition, digital twin development created the core cyber-physical system, and validation confirmed the system's accuracy and practical utility. The resulting digital twin enables virtual testing scenarios, reduces development time, and provides new capabilities for robotics research and industrial applications.

Key contributions include validated methodologies for digital twin creation in robotics, frameworks for real-time synchronization between physical and digital systems, and demonstrated applications of cyber-physical systems in mobile robotics. This work advances the state-of-the-art in digital twin technology and provides a foundation for future research in autonomous systems development.



% Sample content to demonstrate LaTeX command. You will likely delete this line and the
% next \input{sample/*} lines. You are also safe to delete the sample/ folder and its
% content once you refershed your LaTeX skills. Also check the appendix samples.
%\input{sample/1content.tex}
%\input{sample/2lorem.tex}
%\input{sample/3graph.tex}

%----------------------------------------------------------------------------------------
%	BIBLIOGRAPHY REFERENCES
%----------------------------------------------------------------------------------------

\input{style/biblio.tex}

%----------------------------------------------------------------------------------------
%	APPENDICES
%----------------------------------------------------------------------------------------

\input{style/appendix.tex}
%force smaller vertical spacing in table of content
%!!! There can be some fun depending if the appendices have (sub)sections or not :D
% You will have to play with these numbers and eventually add the \vspace line  before
% some \chapter and force another number.
% To add more fun, time to time the table of content get wrong after a build :(
\addtocontents{toc}{\vspace{11pt}}
\pretocmd{\chapter}{\addtocontents{toc}{\protect\vspace{-24pt}}}{}{}

\liite{1}% This is a hack to have right page numbering for each appendix. Make sure to
% use a unique number for each appendix.
% Appendix 1: Jaska Platform Technical Specifications

\chapter{Jaska Platform Technical Specifications}

% Technical specifications for the Jaska mobile robot platform
% Add detailed technical specifications here

\section{Hardware Specifications}

[Content to be added - hardware components, dimensions, weight, etc.]

\section{Sensor Specifications}

[Content to be added - LiDAR, camera, IMU specifications]

\section{Motor and Control System Specifications}

[Content to be added - motor specifications, MODBUS details, etc.]

\section{Software Requirements}

[Content to be added - ROS2 version, dependencies, etc.]
% Jaska Platform Technical Specifications

%\addtocontents{toc}{\vspace{11pt}}%fix vertical space for Table of Content
\liite{2}
% Appendix 2: Source Code Repository Structure

\chapter{Source Code Repository Structure}

% Documentation of the source code organization
% Add repository structure here

\section{Repository Overview}

[Content to be added - main repository structure, organization]

\section{ROS2 Package Structure}

[Content to be added - ROS2 packages, nodes, launch files]

\section{Simulation Files}

[Content to be added - URDF models, world files, configuration files]

\section{Documentation and Scripts}

[Content to be added - build scripts, installation guides, etc.]
% Source Code Repository Structure


%----------------------------------------------------------------------------------------
%	THIS IS THE END
%----------------------------------------------------------------------------------------
\end{document}
