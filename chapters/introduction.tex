% Introduction

\chapter{Introduction}

\section{Background and Context}

The need for advanced autonomous systems is growing quickly, making mobile robotics
an important area of study, especially for tough industrial settings like factories
or mines. This Master's Thesis focuses on two key needs: combining sensors effectively
and creating reliable simulations. These are essential for building and testing
advanced robotic systems that can work on their own. Challenges include dealing with
sensor errors, changing environments, and the need for efficient testing methods.

The main focus is a mobile robot used as a prototype to test autonomous navigation,
sensor integration, and new technologies like custom wheels and improved LiDAR mapping.
A big first step is getting this robot working again so it can be used for experiments.
This involves connecting different sensors—such as LiDAR, cameras, and Inertial
Measurement Units (IMUs)—into one system. This can be hard due to issues like
mismatched data and environmental noise. Simulations are also crucial to test the
robot virtually before real-world use, saving time and money.

This study builds on existing research. Quigley et al. (2009) developed the Robot
Operating System (ROS), a widely used tool that helps combine sensors for autonomous
robots. Thrun et al. (2005) explain how probability can improve location and mapping,
showing the importance of good sensor fusion. Siciliano et al. (2009) stress the
value of simulations to connect theory with practice. These studies highlight gaps
in real-time sensor use and simulation accuracy, which this thesis aims to address.

To do this work, the researcher needs specific skills. A strong background in
mechatronics, automation, robotics, or electrical engineering is necessary. Good
knowledge of programming languages like Python and C++ is required for writing
control codes and sensor programs. Familiarity with robotics tools like ROS/ROS2
is also helpful, as it makes the work easier and fits industry standards.

\section{Thesis Objectives and Scope}

The main goal of this thesis is to create a comprehensive digital twin of a mobile
robot platform through sensor integration and simulation environment development.
The primary focus is on building an accurate digital representation that mirrors
the physical robot's capabilities and behaviors. This involves systematically
integrating sensors and developing simulation models that can effectively replicate
real-world scenarios. The work is divided into three phases, with the digital twin
development serving as the central contribution of this research.

\textbf{Phase 1: Sensor Integration and Data Acquisition}
This foundational phase focuses exclusively on integrating and calibrating the
robot's sensory systems. Tasks include setting up and configuring the camera,
LiDAR, and IMU sensors to ensure proper data collection and synchronization.
This involves solving technical challenges such as connecting the Unitree L2 LiDAR
driver, calibrating sensor parameters, and establishing reliable data streams.
For initial perception validation, the phase includes creating basic maps using
POINT-LIO SLAM to verify sensor functionality. Additionally, this phase involves
comprehensive evaluation of simulation platforms—Gazebo, Webots, CoppeliaSim, and
Isaac Sim—to determine the most suitable environment for digital twin development,
with emphasis on ROS2 compatibility as highlighted by Quigley et al. (2009).

\textbf{Phase 2: Digital Twin Development and Implementation}
This phase represents the core contribution of the thesis, focusing on creating
a comprehensive digital twin of the physical robot. The primary objective is
developing a high-fidelity virtual model that accurately represents the robot's
physical characteristics, sensor configurations, and behavioral dynamics. This
involves creating detailed CAD models, implementing physics-based simulations,
and establishing real-time data synchronization between the physical and digital
systems. The digital twin incorporates all sensor modalities (LiDAR, camera, IMU)
and replicates the robot's kinematic and dynamic properties. Validation occurs
through comparative testing between simulated and real-world sensor data, using
probabilistic methods from Thrun et al. (2005) to ensure accuracy. This phase
establishes the foundation for virtual testing and development workflows.

\textbf{Phase 3: Digital Twin Validation and Applications}
The final phase validates the digital twin through comprehensive testing and
demonstrates its practical applications. This involves implementing ROS2-based
communication frameworks to ensure seamless data exchange between physical and
digital systems, as outlined by Quigley et al. (2009). The digital twin is tested
for accuracy in various scenarios, including sensor data comparison, kinematic
validation, and environmental perception tasks. Applications include virtual
testing of navigation algorithms, sensor fusion validation, and predictive
maintenance capabilities. The phase culminates in demonstrating the digital twin's
effectiveness for development, testing, and operational support, establishing its
value for future research and industrial applications as supported by concepts
from Siciliano et al. (2009).

\section{Expected Outcomes and Contributions}

Completing this thesis will provide significant contributions to mobile robotics
and digital twin technology. First, it will deliver a validated methodology for
creating high-fidelity digital twins of mobile robotic platforms, establishing
best practices for sensor integration and virtual modeling. Second, it will
provide a comprehensive digital twin system that enables virtual testing,
development, and validation workflows, addressing the critical need for reliable
simulation environments as highlighted by Siciliano et al. (2009). Third, the
research will contribute validated frameworks for real-time synchronization
between physical and digital systems, advancing the state-of-the-art in
cyber-physical systems for robotics as supported by methodologies from
Thrun et al. (2005). These achievements will not only advance digital twin
technology but also provide valuable tools and knowledge for the broader
mobile robotics research community.
