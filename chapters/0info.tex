\documentclass[12pt,a4paper,oneside,article]{memoir}%Do not touch this first line ;)

% Global information (title of your thesis, your name, degree programme, major, etc.)

\def\bilingual{no}%For international student writing in English, only one language and one abstract.
%\def\bilingual{yes}%For Finnish students, you must have 2 abstracts, one in English and one in your native language (Finnish or Swedish), so "yes", your thesis is bilingual.

\def\thesislang{english} % "english" is the only other supported language currently. If someone has the swedish, please contribute!
%\def\thesislang{finnish} %change this depending on the main language of the thesis.

\def\secondlang{english} %if the main language is Finnish (or Swedish), you must have 2 abstracts (one in Finnish (or Swedish) and one in English)
%\def\secondlang{finnish}
%If the main language is English and that you are native Finnish (or Swedish) speaker, you must have also abstract in your native language on top of the English one.

\author{Hai To} %your first name and last name

%\def\alaotsikko{Alaotsikko/Subtitle} %DISABLED, seems not to be an option with the 2018 template. If you really need it, uncomment and modify style/title.tex accordingly.

%License
%When publishing your thesis to theseus.fi, you can keep all rights reserved to you,
%or use one of the Creative Commons https://creativecommons.org/licenses/?lang=en
%This attribute will set the license in the metadata of the generated pdf.
%possible options (case sensitive):
%all (keep all rights reserved to yourself)
%by (Attribution)
%by-sa (Attribution-ShareAlike)
%by-nd (Attribution-NoDerivs)
%by-nc (Attribution-NonCommercial)
%by-nc-sa (Attribution-NonCommercial-ShareAlike)
%by-nc-nd (Attribution-NonCommercial-NoDerivs)
%Note that theseus do not accept dedication to public domain CC0
\def\thesiscopy{all}

%Finnish section, for title/abstract
\def\otsikko{Sensorien ja simulaatioympäristön integrointi liikkuvalle robottialustalle}
\def\tutkinto{Insinööri (YAMK)} % Master of Engineering
\def\kohjelma{Tieto- ja viestintätekniikka}
\def\suuntautumis{Robotiikka ja automaatio}
\def\thesisfi{Opinnäytetyö}
\def\ohjaajat{
Markku Niiranen, Yliopettaja\newline
}
\def\tiivistelma{
Opinnäytetyö keskittyy itsenäisesti toimivan liikkuvan robotin rakentamiseen ja testaamiseen. Se sisältää LiDAR-, kamera- ja IMU-sensoreiden lisäämisen sekä simulaation luomisen robotin testaamiseksi ennen todellista käyttöä. Työ jakautuu kolmeen vaiheeseen: laitteiston asettamiseen, digitaalisen mallin tekemiseen ja autonomian testaamiseen todellisessa maailmassa. \newline

Tutkimus käyttää ROS2:ta sensoreiden ja navigaation hallintaan ja arvioi työkaluja kuten Gazebo ja Isaac Sim parhaan simulaation valitsemiseksi. Odotetut tulokset sisältävät toimivan robotin, valitun simulaatiotyökalun ja perusautonomiset navigointijärjestelmän.
}
\def\avainsanat{Liikkuva robotti, Sensorien integrointi, Simulaatio, ROS2, Autonominen navigointi}
\def\aihe{Sensorien ja simulaatioympäristön integrointi autonomisen liikkuvan robottialustan kehitykseen ja testaukseen.}%for the pdf metadata/properties. If not used, empty it and also the \def\subject.

%English section, for title/abstract
\title{Integration of Sensors and Simulation Environment for a Mobile Robot Platform}
\def\metropoliadegree{Master of Engineering} % change to your needs, e.g. "master", etc.
\def\metropoliadegreeprogramme{Information Technology}
\def\metropoliaspecialisation{Robotics and Automation}
\def\thesisen{Master's Thesis} % change to your need, e.g. master's
\def\metropoliainstructors{
Markku Niiranen, Principal Lecturer\newline
}
\def\abstract{
The thesis focuses on building and testing a mobile robot that can work on its own. It involves adding sensors like LiDAR, cameras, and IMUs, and creating a simulation to test the robot before real use. The work is divided into three steps: setting up the hardware, making a digital model, and testing autonomy in the real world. \newline

The study uses ROS2 to manage sensors and navigation, and reviews tools like Gazebo and Isaac Sim to pick the best simulation. Expected results include a working robot, a chosen simulation tool, and a basic autonomous navigation system. This work helps improve robot technology for tough industrial settings.
}
\def\metropoliakeywords{Mobile Robot, Sensor Integration, Simulation, ROS2, Autonomous Navigation}
\def\subject{Integration of sensors and simulation environment for autonomous mobile robot platform development and testing.}%for the pdf metadata/properties. If not used, empty it and also the \def\aihe.his first line ;)
