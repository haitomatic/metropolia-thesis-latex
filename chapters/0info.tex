\documentclass[12pt,a4paper,oneside,article]{memoir}%Do not touch this first line ;)

% Global information (title of your thesis, your name, degree programme, major, etc.)

\def\bilingual{no}%For international student writing in English, only one language and one abstract.
%\def\bilingual{yes}%For Finnish students, you must have 2 abstracts, one in English and one in your native language (Finnish or Swedish), so "yes", your thesis is bilingual.

\def\thesislang{english} % "english" is the only other supported language currently. If someone has the swedish, please contribute!
%\def\thesislang{finnish} %change this depending on the main language of the thesis.

\def\secondlang{english} %if the main language is Finnish (or Swedish), you must have 2 abstracts (one in Finnish (or Swedish) and one in English)
%\def\secondlang{finnish}
%If the main language is English and that you are native Finnish (or Swedish) speaker, you must have also abstract in your native language on top of the English one.

\author{Hai To} %your first name and last name

%\def\alaotsikko{Alaotsikko/Subtitle} %DISABLED, seems not to be an option with the 2018 template. If you really need it, uncomment and modify style/title.tex accordingly.

%License
%When publishing your thesis to theseus.fi, you can keep all rights reserved to you,
%or use one of the Creative Commons https://creativecommons.org/licenses/?lang=en
%This attribute will set the license in the metadata of the generated pdf.
%possible options (case sensitive):
%all (keep all rights reserved to yourself)
%by (Attribution)
%by-sa (Attribution-ShareAlike)
%by-nd (Attribution-NoDerivs)
%by-nc (Attribution-NonCommercial)
%by-nc-sa (Attribution-NonCommercial-ShareAlike)
%by-nc-nd (Attribution-NonCommercial-NoDerivs)
%Note that theseus do not accept dedication to public domain CC0
\def\thesiscopy{all}

%Finnish section, for title/abstract
\def\otsikko{Sensorien ja simulaatioympäristön integrointi liikkuvalle robottialustalle}
\def\tutkinto{Insinööri (YAMK)} % Master of Engineering
\def\kohjelma{Tieto- ja viestintätekniikka}
\def\suuntautumis{Verkot ja palvelut}
\def\thesisfi{Opinnäytetyö}
\def\ohjaajat{
Markku Niiranen, Yliopettaja\newline
}
\def\tiivistelma{
Opinnäytetyö keskittyy kattavan digitaalisen kaksosen kehittämiseen liikkuvalle robottialustalle sensorien integroinnin ja simulaatioympäristön kautta. Työn pääpaino on luoda tarkka digitaalinen malli, joka heijastaa fyysisen robotin ominaisuuksia ja käyttäytymistä. Työ jakautuu kolmeen vaiheeseen: sensorien integrointi, digitaalisen kaksosen kehittäminen ja validointi. \newline

Tutkimus käyttää ROS2:ta reaaliaikaisen tiedonsiirron hallintaan fyysisen ja digitaalisen järjestelmän välillä. Digitaalinen kaksonen sisältää LiDAR-, kamera- ja IMU-sensorit sekä fysiikkapohjaiset simulaatiot. Odotetut tulokset sisältävät validoidun metodologian digitaalisten kaksosten luomiseen, toimivan cyber-fyysisen järjestelmän ja virtuaalisen testausympäristön robotiikkatutkimukselle.
}
\def\avainsanat{Digitaalinen kaksonen, Sensorien integrointi, Simulaatio, ROS2, Cyber-fyysinen järjestelmä}
\def\aihe{Digitaalisen kaksosen kehittäminen liikkuvalle robottialustalle sensorien integroinnin ja simulaatioympäristön kautta.}%for the pdf metadata/properties. If not used, empty it and also the \def\subject.

%English section, for title/abstract
\title{Integration of Sensors and Simulation Environment for a Mobile Robot Platform}
\def\metropoliadegree{Master of Engineering} % change to your needs, e.g. "master", etc.
\def\metropoliadegreeprogramme{Information Technology}
\def\metropoliaspecialisation{Networking and Services}
\def\thesisen{Master's Thesis} % change to your need, e.g. master's
\def\metropoliainstructors{
Markku Niiranen, Principal Lecturer\newline
}
\def\abstract{
The thesis focuses on developing a comprehensive digital twin for a mobile robot platform through sensor integration and simulation environment development. The primary objective is creating an accurate digital representation that mirrors the physical robot's capabilities and behaviors. The work is divided into three phases: sensor integration, digital twin development, and validation. \newline

The study utilizes ROS2 for real-time communication between physical and digital systems. The digital twin incorporates LiDAR, camera, and IMU sensors with physics-based simulations to ensure high fidelity. Expected outcomes include a validated methodology for digital twin creation, a functioning cyber-physical system, and a virtual testing environment that advances mobile robotics research and development.
}
\def\metropoliakeywords{Digital Twin, Sensor Integration, Simulation, ROS2, Cyber-Physical Systems}
\def\subject{Digital twin development for mobile robot platform through sensor integration and simulation environment.}%for the pdf metadata/properties. If not used, empty it and also the \def\aihe.his first line ;)
