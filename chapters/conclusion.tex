% Conclusion
%\clearpage%if the chapter heading starts close to bottom of the page, force a line break and remove the vertical vspace
\vspace{21.5pt}
\chapter{Conclusion}

This thesis successfully developed a comprehensive digital twin system for a mobile robot platform, achieving the primary objective of creating an accurate digital representation through systematic sensor integration and simulation environment implementation. The research demonstrated that high-fidelity digital twins can be effectively created for mobile robotic systems, providing valuable tools for virtual testing, development, and validation.

The three-phase approach proved effective: sensor integration established the foundation for data acquisition, digital twin development created the core cyber-physical system, and validation confirmed the system's accuracy and practical utility. The resulting digital twin enables virtual testing scenarios, reduces development time, and provides new capabilities for robotics research and industrial applications.

Key contributions include validated methodologies for digital twin creation in robotics, frameworks for real-time synchronization between physical and digital systems, and demonstrated applications of cyber-physical systems in mobile robotics. This work advances the state-of-the-art in digital twin technology and provides a foundation for future research in autonomous systems development.

